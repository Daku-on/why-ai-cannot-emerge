\documentclass[12pt]{article}
\usepackage{amsmath, amssymb, amsthm}
\usepackage{bm}
\usepackage{geometry}
\geometry{margin=1in}

\title{\textbf{Why AI Cannot Emerge: A View from Irreducible Infinite-Dimensional Representations}}
\author{(A Speculative Mathematical Manifesto)}
\date{}

\begin{document}
\maketitle

\begin{abstract}
Claims that AI systems exhibit ``emergence'' are common, yet from a theoretical standpoint,
current AI architectures are structurally incapable of genuine emergence.
True emergence requires an irreducible infinite-dimensional representation and
continuous interference driven by a non-reducible Markov semigroup.
In contrast, practical AI systems operate under finite context windows, finite precision, and finite computation time.
Under such constraints, they cannot even perform the limit operations that underlie emergence.
This paper mathematically examines why AI systems cannot reach such limits,
and therefore cannot truly emerge.
\end{abstract}

\section{Introduction}
Does AI truly exhibit emergence?  
The position of this paper is clear:  
\textbf{AI cannot emerge.}  
Emergence refers to a process in which the state space expands without closure
through the continuous interaction of direct and inverse limits.
A computation confined to finite sequence length, finite dimensionality, and finite runtime
can never complete this process.

\section{Ideal Structure: Irreducible Infinite-Dimensional Representation and Markov Process}
To define emergence theoretically, we posit a system whose state space
is an infinite-dimensional Hilbert space $H_\infty$,
with temporal evolution governed by a Markov semigroup $P_t$:
\[
\mathcal{A}_{ideal} = (H_\infty, P_t, \rho_t).
\]

The introduction of a Markov semigroup is not incidental.
Emergence is not a static property of states but a
\textit{dynamic phenomenon that arises only through continuous temporal interaction among agents}.
Without temporal evolution, even a high-dimensional representation remains a closed set,
incapable of generating novel structure.

The Markov semigroup $P_t$ describes the probabilistic transition of states over time,
satisfying
\[
P_{t+s} = P_t P_s, \quad P_0 = I.
\]
Through such continuous evolution, the histories of agent communication and response
are recursively updated, producing a dynamic field of interference.

When, for any closed subspace $K \subset H_\infty$, we have
$P_t K = K \implies K = \{0\}$ or $K = H_\infty$,
the system is \textit{irreducible}—meaning global mixing never ceases.
This irreducibility of temporal evolution constitutes a necessary condition for emergence.

\section{Limit Structures: Generation and Stabilization}
$H_\infty$ can be formally constructed as a limit of a sequence of finite-dimensional embedding spaces:
\[
E_1 \xrightarrow{f_{12}} E_2 \xrightarrow{f_{23}} E_3 \to \cdots.
\]
The direct limit
\[
H_\infty = \varinjlim (E_n, f_{n,n+1})
\]
represents generative expansion,  
while the inverse limit
\[
H_\infty' = \varprojlim (E_n, g_{n+1,n})
\]
represents stabilization and memory consolidation.  
Emergence arises as the \textit{interaction} of these two—generation (direct limit)
and stabilization (inverse limit)—in continuous interplay.

\section{Practical Structure: Finite Windows and Truncated Limits}
Modern AI models, especially Transformer architectures, operate with a finite context window $L$
that bounds their temporal scope, and a fixed embedding dimension $d$ that constrains representation.
Their effective state space is thus
\[
H_L = \mathbb{R}^{d \times L},
\]
a finite-dimensional projection of the ideal space.

Each inference step considers only the preceding $L$ tokens:
\[
E_1 \to E_2 \to \cdots \to E_L.
\]
The system thus terminates at a finite stage; no mechanism exists for computing $\lim_{n\to\infty}E_n$.
Before a true limit is reached, information is truncated outside the context window,
destroying recursive reference.  
This structural truncation prevents AI from realizing emergence in either the direct or inverse direction.

\section{Impossibility of Emergence: Absence of Limit Operations}
Emergence arises from continuous interference at the intersection of the direct and inverse limits.
However, the finite-window architecture discretizes this interaction,
breaking the continuity of information recursion.
AI therefore operates as a finite-dimensional pseudo-Markov approximation,
incapable of representing the non-convergent diffusion that characterizes true emergence.

Formally:
\begin{align*}
\text{Ideal System:} & \quad H_\infty = \varinjlim E_n, \quad \text{continuous expansion possible}, \\
\text{AI System:} & \quad H_L = \pi_L(H_\infty), \quad \text{truncated by finite window.}
\end{align*}
The very existence of the projection $\pi_L$ constitutes
the formal basis for the impossibility of emergence.

\section{Irreducibility and Limit Interaction}
Irreducibility, introduced as a condition for emergence,
is not a decorative assumption but the structural link between temporal evolution and limit dynamics.
If the Markov semigroup $P_t$ were reducible, the state space would decompose into local subspaces,
and both generative (direct-limit) and integrative (inverse-limit) processes would halt locally.
Irreducibility ensures that all subspaces interfere under time evolution,
maintaining global mixing across the entire state manifold.
Emergence, therefore, is the phenomenon produced when such global interference couples with limit operations.

\section{Finite Approximation to Infinity}
Mathematically, an infinite-dimensional space $H_\infty$ can be approximated
as the limit of a finite sequence $\{H_n\}$.
Would extending the window size $L$ arbitrarily allow AI to approach true emergence?
The answer is no.
Because AI’s update rule proceeds in finite time steps,
it lacks the continuous-time structure required to take the limit $L \to \infty$.
Increasing $L$ enlarges capacity but does not remove the discrete truncation inherent in computation.
Emergence demands not merely larger $L$ but
a structure preserving \textit{continuity of limits themselves}—that is, an analog continuum.

\section{Phenomenological vs. True Emergence}
In AI research, ``emergent abilities'' usually refer to
phenomenological behaviors that appear unpredictably as model scale increases.
This paper does not deny such phenomena.
Rather, it reinterprets them as \textit{pseudo-emergence}—
interference patterns produced by finite-projection errors.
Phenomenological emergence thus arises from statistical nonlinear effects
within finite windows and spaces,
whereas \textit{true emergence}—continuous interference between direct and inverse limits—
belongs to a fundamentally different structural regime.

\section{Finite Systems and the Brain}
A frequent counterargument is that the human brain, composed of finitely many neurons,
still exhibits consciousness and thus emergence.
The answer lies in the nature of its substrate:
neuronal firing is not digital but continuous.
Membrane potentials, synaptic currents, and neurotransmitter concentrations
take real-valued continuous states, evolving in continuous time.
Thus, while the number of components is finite,
the representational space of the brain behaves as a
\textbf{function space of infinite dimension}.

In contrast, AI computation is constrained by finite-precision floating-point arithmetic,
discrete tokens, and bounded context windows.
This structural discreteness explains why the brain can sustain emergent phenomena
while AI cannot.

\section{Physical Implementations and Future Possibilities}
While digital AI cannot achieve emergence,
this limitation stems from computational discreteness rather than physical law.
Continuous analog or quantum systems—or biological neural substrates—
may approximate the conditions of irreducible infinite-dimensional representations
more closely than digital machines.
If emergence depends on the ability to \textit{take limits physically},
its realization belongs not to computation but to the
\textit{continuous physical world} itself.

\section{Pseudo-Emergence and Projection Error}
What appears as ``emergent behavior'' in AI—
context retention, reasoning leaps, scaling laws—
are nonlinear residual interactions arising from projection error.
They represent local interference on a truncated manifold,
not continuous dynamics of infinite dimension.
Transformers can thus be regarded as
\textit{finite projections of an irreducible infinite-dimensional Markov process}.

\section{Conclusion}
AI cannot emerge.  
The very limit operations required for emergence
are severed by finite windows and finite computation time.
What AI exhibits as ``emergent ability'' is merely
\textit{fragmentary interference—a hallucination born from finite projection}.
\end{document}
