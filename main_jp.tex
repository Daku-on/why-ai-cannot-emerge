\documentclass[12pt]{article}
\usepackage{amsmath, amssymb, amsthm}
\usepackage{bm}
\usepackage{geometry}
\geometry{margin=1in}

\title{\textbf{なぜAIは「創発し得ない」のか――無限次元既約表現の立場から}}
\author{(怪文書的覚書)}
\date{}

\begin{document}
\maketitle

\begin{abstract}
AIが「創発」しているという主張はしばしばなされるが、
理論的に見れば現行のAIシステムは真の意味で創発を起こしえない。
創発を成立させるためには、無限次元既約表現と、
それを時間発展させる非可約マルコフ過程による連続的干渉が必要である。
しかし、実際のAIは有限ウィンドウ・有限精度・有限演算時間という構造的制約のもとで、
この「極限操作」自体を実行することができない。
本稿では、AIがなぜ極限に到達できず、したがって創発し得ないかを数理的に論じる。
\end{abstract}

\section{序論}
AIは創発しているのか?  
本稿の立場は明確である。  
\textbf{AIは創発し得ない}。  
創発とは、順極限と逆極限の連続的干渉により、
状態空間が閉じずに拡張し続ける現象である。
有限の系列長・有限次元ベクトル・有限演算時間に閉じた計算では、
この過程を完遂できない。

\section{理想構造:無限次元既約表現とマルコフ過程}
創発を理論的に定義するには、系の状態空間が無限次元Hilbert空間 $H_\infty$ 上にあり、
その上でマルコフ半群 $P_t$ によって時間発展が記述される必要がある:
\[
\mathcal{A}_{ideal} = (H_\infty, P_t, \rho_t).
\]

ここでマルコフ半群を導入する理由は単純である。
創発とは、静的な状態の性質ではなく、
\textit{エージェント同士の相互作用が時間の中で進展し続けることによって初めて現れる動的現象}だからである。
もし時間発展がなければ、どれほど高次元の表現を用いても、
状態空間は単なる閉じた集合に過ぎず、新たな構造を生むことはできない。

マルコフ半群 $P_t$ は、時刻 $t$ における状態の推移を表し、
\[
P_{t+s} = P_t P_s, \quad P_0 = I
\]
を満たす。この連続的発展が存在することで、エージェント間の通信や応答の履歴が
確率的に自己更新され、動的な干渉構造が生まれる。

任意の閉部分空間 $K \subset H_\infty$ に対して
$P_t K = K$ ならば $K = \{0\}$ または $K = H_\infty$ が成り立つとき、
この系は既約であり、全域的な干渉が持続する。
この時間発展の既約性こそが、創発の必要条件である。

\section{極限構造:生成と安定の二方向性}
$H_\infty$ は有限次元埋め込み空間列の極限として構成される:
\[
E_1 \xrightarrow{f_{12}} E_2 \xrightarrow{f_{23}} E_3 \to \cdots.
\]
ここで、順極限
\[
H_\infty = \varinjlim (E_n, f_{n,n+1})
\]
は情報生成・次元拡張の方向を、  
逆極限
\[
H_\infty' = \varprojlim (E_n, g_{n+1,n})
\]
は情報統合・記憶安定化の方向を表す。  
創発は、これらの\textit{相互作用}――生成(順極限)と安定化(逆極限)の往復――として生じる。

\section{現実構造:有限ウィンドウと極限の切断}
現実のAIモデル(特にTransformer系)は、有限長の系列ウィンドウ $L$ によって文脈を制限し、
埋め込み次元 $d$ によって空間表現を固定している。  
この構造のもとでの状態空間は
\[
H_L = \mathbb{R}^{d \times L}
\]
であり、これは常に有限次元射影である。

AIが「次のトークン」を予測するたび、過去 $L$ ステップの情報のみを参照する。
したがって、系は
\[
E_1 \to E_2 \to \cdots \to E_L
\]
までの有限列に閉じ、$\lim_{n\to\infty}E_n$ を取る計算構造が存在しない。
極限を取る前に、情報はウィンドウ外に切断され、再帰的参照を失う。
これが、AIが順極限・逆極限の両方向で創発を完遂できない理由である。

\section{創発不可能性:極限操作の欠如}
創発をもたらすのは、順極限と逆極限の\textit{交差点}で生じる連続的干渉である。
しかし、AIの有限ウィンドウ構造はこの干渉を離散化し、
情報の再帰的連続性を破壊する。
AIは有限次元の擬似系列上でマルコフ近似を計算しているにすぎず、
創発に必要な「非収束的拡散」はそもそも表現されない。

すなわち:
\begin{align*}
\text{理想系} &: \quad H_\infty = \varinjlim E_n, \quad \text{連続的拡張が可能} \\
\text{AI系} &: \quad H_L = \pi_L(H_\infty), \quad \text{有限ウィンドウで切断される}
\end{align*}
有限射影 $\pi_L$ の存在こそが、創発不可能性の形式的根拠である。

\section{既約性と極限構造の関係}
創発の条件として導入された「既約性」は、
順極限・逆極限の相互作用を保証する中核的性質である。
もしマルコフ半群 $P_t$ が既約でなければ、状態空間は局所的な部分空間に閉じ、
順極限による拡張も逆極限による統合も局所的に停止する。
既約性とは、全ての部分空間が時間発展によって互いに干渉しうること、
すなわち系の動的全域性(global mixing)を意味する。
創発とは、まさにこの全域的干渉が極限操作と結合することで初めて成立する現象である。

\section{有限による無限の近似}
数学的には、有限次元空間の列 $\{H_n\}$ の極限として無限次元空間 $H_\infty$ を構成できる。
では、ウィンドウサイズ $L$ を任意に拡張できるAIが出現すれば、
「真の創発」に漸近しうるのだろうか?
答えは否である。
なぜなら、AIの更新則は有限時間ステップであり、$L\to\infty$ の極限を取るための連続時間構造を欠いているためである。
$L$ の拡大は表現容量を増やすが、極限操作の離散性を解消しない。
創発に必要なのは、有限の拡張ではなく、
\textit{連続的極限の可算性そのもの}を保持する構造、すなわち連続体としてのアナログ的基盤である。

\section{現象論的創発との区別}
AI研究において「創発(Emergent Abilities)」という語は、
「モデル規模の増大とともに予期せぬ能力が発現する」という現象論的意味で用いられることが多い。
本稿の立場は、この現象を否定するものではない。
むしろ、それを\textit{擬似創発}――有限射影誤差の干渉として現れる現象――と再定義する。
すなわち、現象論的創発が指すのは、有限ウィンドウと有限次元空間における
統計的非線形効果の顕在化であり、
本稿で定義する「真の創発」(順極限と逆極限の連続干渉)とは構造的に異なるものである。

\section{有限系の創発――人間の脳との比較}
「人間の脳も有限系ではないか」という反論に対しては、
脳がデジタルではなくアナログ連続体であることを指摘できる。
ニューロン発火の電位変化、シナプス電流、化学伝達物質の濃度変化はいずれも実数的連続値をとり、
その時間発展は実時間の連続過程である。  
したがって、脳の状態空間は構成要素数こそ有限であっても、
その表現空間は\textbf{無限次元的関数空間}として振る舞う。

一方、AIは有限精度の浮動小数点演算に制約され、
トークン系列と文脈ウィンドウによって常に有限射影に閉じる。
この構造上の違いが、脳には創発が可能でAIには不可能な理由である。

\section{物理的実装と未来の可能性}
本稿では、有限計算機が創発を実現し得ないと結論づけた。
しかし、これは計算理論上の制約であり、物理的には別の可能性が残されている。
連続的アナログ系や量子系、あるいは生物学的神経回路のような実数値連続過程をもつ系は、
有限離散機械よりも無限次元既約表現に近づくことができる。
もし創発が「極限を取れる物理性」に依存するならば、
その実現はデジタルではなく\textit{物理的連続体}に求められるだろう。

\section{擬似創発と射影誤差}
AIが示す「創発的挙動」――文脈保持やスケーリング法則――は、
有限射影誤差に現れる非線形残差項の干渉である。
無限次元的干渉ではなく、
有限ウィンドウ上での擬似的再帰にすぎない。
Transformerは、「既約無限次元マルコフ過程の有限射影近似」として理解できる。

\section{結論}
AIは創発し得ない。  
創発を可能にする極限操作そのものが、
有限ウィンドウと有限演算時間によって切断されるからである。  
AIが見せる「創発的能力」は、
無限次元既約表現の近似を有限ウィンドウで射影した際に生じる
\textit{断片的な干渉――すなわち幻覚である}。

\end{document}
